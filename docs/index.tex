% Options for packages loaded elsewhere
\PassOptionsToPackage{unicode}{hyperref}
\PassOptionsToPackage{hyphens}{url}
%
\documentclass[
]{article}
\usepackage{amsmath,amssymb}
\usepackage{iftex}
\ifPDFTeX
  \usepackage[T1]{fontenc}
  \usepackage[utf8]{inputenc}
  \usepackage{textcomp} % provide euro and other symbols
\else % if luatex or xetex
  \usepackage{unicode-math} % this also loads fontspec
  \defaultfontfeatures{Scale=MatchLowercase}
  \defaultfontfeatures[\rmfamily]{Ligatures=TeX,Scale=1}
\fi
\usepackage{lmodern}
\ifPDFTeX\else
  % xetex/luatex font selection
\fi
% Use upquote if available, for straight quotes in verbatim environments
\IfFileExists{upquote.sty}{\usepackage{upquote}}{}
\IfFileExists{microtype.sty}{% use microtype if available
  \usepackage[]{microtype}
  \UseMicrotypeSet[protrusion]{basicmath} % disable protrusion for tt fonts
}{}
\makeatletter
\@ifundefined{KOMAClassName}{% if non-KOMA class
  \IfFileExists{parskip.sty}{%
    \usepackage{parskip}
  }{% else
    \setlength{\parindent}{0pt}
    \setlength{\parskip}{6pt plus 2pt minus 1pt}}
}{% if KOMA class
  \KOMAoptions{parskip=half}}
\makeatother
\usepackage{xcolor}
\usepackage[margin=1in]{geometry}
\usepackage{graphicx}
\makeatletter
\def\maxwidth{\ifdim\Gin@nat@width>\linewidth\linewidth\else\Gin@nat@width\fi}
\def\maxheight{\ifdim\Gin@nat@height>\textheight\textheight\else\Gin@nat@height\fi}
\makeatother
% Scale images if necessary, so that they will not overflow the page
% margins by default, and it is still possible to overwrite the defaults
% using explicit options in \includegraphics[width, height, ...]{}
\setkeys{Gin}{width=\maxwidth,height=\maxheight,keepaspectratio}
% Set default figure placement to htbp
\makeatletter
\def\fps@figure{htbp}
\makeatother
\setlength{\emergencystretch}{3em} % prevent overfull lines
\providecommand{\tightlist}{%
  \setlength{\itemsep}{0pt}\setlength{\parskip}{0pt}}
\setcounter{secnumdepth}{-\maxdimen} % remove section numbering
\ifLuaTeX
  \usepackage{selnolig}  % disable illegal ligatures
\fi
\IfFileExists{bookmark.sty}{\usepackage{bookmark}}{\usepackage{hyperref}}
\IfFileExists{xurl.sty}{\usepackage{xurl}}{} % add URL line breaks if available
\urlstyle{same}
\hypersetup{
  pdftitle={임상 데이터 의학 (Clinical Data Medicine with R)},
  pdfauthor={홍윤호},
  hidelinks,
  pdfcreator={LaTeX via pandoc}}

\title{임상 데이터 의학 (Clinical Data Medicine with R)}
\author{홍윤호}
\date{}

\begin{document}
\maketitle

\hypertarget{uxbaa9uxd45cuxc640-uxb0b4uxc6a9}{%
\subsubsection{목표와 내용}\label{uxbaa9uxd45cuxc640-uxb0b4uxc6a9}}

\begin{itemize}
\tightlist
\item
  R의 기본 문법을 이해하고, 데이터 정제와 가공, 시각화를 할 수 있다.~
\item
  통계적 검정과 추정을 이해하고, 데이터 분석에 적용할 수 있다.~
\item
  기계학습의 기본 개념과 작업 과정을 설명할 수 있다.~
\item
  기계학습의 주요 알고리즘을 이해하고, 분류(classification)와
  회귀(regression), 비지도 기계학습에 적용할 수 있다.~
\item
  협업과 재현가능한 연구(reproducible research)를 위한 데이터 과학의
  도구들을 활용할 수 있다(Git/Github, Rmarkdown/Quarto, Shiny).~
\item
  데이터 과학이 의학 연구와 의료의 디지털 전환에 어떻게 활용되고 기여할
  수 있는지 이해한다.~
\end{itemize}

\hypertarget{uxad50uxc7ac}{%
\subsubsection{교재}\label{uxad50uxc7ac}}

\begin{itemize}
\tightlist
\item
  \href{https://www-bcf.usc.edu/~gareth/ISL/}{An Introduction to
  Statistical Learning (with Applications in R), Springer (Gareth James,
  Daniel Witten, Trevor Hastie, Robert Tibshirani, 2013)}
\end{itemize}

\hypertarget{uxac15uxc758uxb85d}{%
\subsubsection{강의록}\label{uxac15uxc758uxb85d}}

\begin{itemize}
\tightlist
\item
  \href{Intro.html}{데이터 과학과 의료}~
\item
  \href{R_basics.html}{R 프로그래밍의 기초}~
\item
  \href{Preprocessing.html}{데이터 전처리(Data preprocessing)}~
\item
  \href{Visualization.html}{시각화(Visualization)}~
\item
  \href{ML_basics.html}{통계적 모델링과 기계학습의 기초}~
\item
  \href{Bootstrapping.html}{부트스트래핑(Bootstrapping)}~
\item
  \href{LinearRegression.html}{선형회귀 모델(Linear regression model)}~
\item
  \href{LME.html}{선형혼합효과 모델(Linear mixed effects model)}~
\item
  \href{LogisticRegression.html}{로지스틱회귀 모델(Logistic regression
  model)}~
\item
  \href{LDA.html}{선형판별분석(Linear discriminant analysis)}~
\item
  \href{ROC.html}{ROC curve}~
\item
  \href{D7_CV_regularization.html}{Class 7. 교차검증과
  정규화(Ridge/Lasso)}~
\item
  \href{RandomForest_GradientBoosting.html}{의사결정나무와 랜덤
  포레스트, Gradient Boosting}~
\item
  \href{D9_Unsupervised.html}{Class 9. 비지도기계학습(PCA, 클러스터링)}~
\end{itemize}

\hypertarget{uxc2e4uxc2b5}{%
\subsubsection{실습}\label{uxc2e4uxc2b5}}

\begin{itemize}
\tightlist
\item
  \href{SDG.html}{Sustainable Development Goals}~
\item
  \href{Classification_problem.html}{분류}~
\item
  \href{breast_cancer.html}{Wisconsin breast cancer dataset}~
\item
  \href{diabetes.html}{Pima indians diabetes dataset}
\item
  \href{PCA.html}{Principal component analysis}~
\item
  \href{sampleSize.html}{Sample size calculation}~
\item
  \href{RandomForest.html}{Random forest \& gradient boosting}.
\end{itemize}

\hypertarget{uxac15uxc88c}{%
\subsubsection{강좌}\label{uxac15uxc88c}}

\textbf{서울대학교 의과대학 2학년 2학기}\\
- 2023년8월2일-9월20일(8주)\\
- 2022년8월2일-9월20일(8주)\\
- 2019년8월6일-10월1일(8주)

\end{document}
